\documentclass{article}
\usepackage[letterpaper,hmargin=1in,vmargin=1.25in]{geometry}
\usepackage{amsmath}
\usepackage{amssymb}
\usepackage{amsthm}

%%% Formatting Notes
% This package allows for block paragraphs - no need for \\ or \noindent
\usepackage{parskip}  

\title{Homework 2}
\author{}
\date{\today}



\begin{document}
\maketitle

These are notes for the directed study on Fourier Analysis from Stein
and Shakarchi \cite{Stein:2003:FA}.


\textbf{Exercises:} Complete 2, 4, 5, 6, 8

\section*{2.)}

\newpage
\section*{4.)}

\newpage
\section*{5.)}

\newpage
\section*{6.)}

\newpage
\section*{8.)}
\emph{Verify that $\frac{1}{2i}\sum_{n\neq 0} \frac{e^{inx}}{n}$ is the Fourier
series of the $2 \pi$-periodic sawtooth function, defined by $f(0) = 0$, and \begin{align*}
    f(x) = \begin{cases}
    - \frac{\pi}{2} - \frac{x}{2} & \text{  if } -\pi < x < 0, \\ 
    \frac{\pi}{2} - \frac{x}{2} & \text{  if } 0 < x < \pi. \\ 
    \end{cases}
\end{align*}
Note that this function is not continuous. Show that nevertheless, the series
converges for every $x$ (by which we mean, as usual, that the symmetric partial
sums of the series converge). In particular, the value of the series at the origin,
namely 0, is the average of the values of $f(x)$ as $x$ approaches the origin from
the left and the right.}

\bibliographystyle{siam}
\bibliography{bibfile2}
\end{document}
